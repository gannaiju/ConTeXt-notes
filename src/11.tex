\chapter{学一点 \MetaPost}

想必你已迫不及待想学习 \MetaPost\ 了。这大概是来自人类上古基因的冲动。人类先学会的是绘画,而后才是文字。只是不要妄图通过这区区一章内容掌握 \MetaPost,因为关于它的全部内容,足够写一本至少三百多页的书籍了。不过,本章内容足以给你打开一扇窗户,让 \MetaPost\ 的优雅气息拂过时常过于严肃的 \ConTeXt\ 世界。

\section{作图环境}

\MetaPost\ 是一种计算机作图语言,与 \TeX\ 一样,皆为宏编程语言。使用 \MetaPost\ 语言编写的代码可被 mpost 程序编译成 PS 格式的图形文件。自 \LuaTeX\ 开始,mpost 的核心功能集成到了 \LuaTeX\ 中,从此以后,在 \TeX\ 环境中使用 \MetaPost\ 语言作图便不需依赖外部程序了。

\ConTeXt\ 为 \MetaPost\ 代码提供了五种环境:

\starttyping
\startMPcode ... \stopMPcode
\startMPpage ... \stopMPpage
\startuseMPgraphic{name} ... \stopuseMPgraphic
\startuniqueMPgraphic{name} ... \stopuniqueMPgraphic
\startreusableMPgraphic{name} ... \stopreusableMPgraphic
\stoptyping

\noindent 第一种环境用于临时作图,生成的图形会被插入到代码所在位置。第二种环境是生成单独的图形文件,以作其他用途。后面三种环境,生成的图形可根据环境的名称作为文章插图随处使用,但它们又有三种不同用途:

\startitemize[nowhite]
\item useMPgraphic:每被使用一次,对应的 \MetaPost\ 代码便会被重新编译一次。
\item uniqueMPgraphic:只要图形所处环境不变,\MetaPost\ 代码只会被编译一次。
\item reusableMPgraphic:无论如何使用,其 \MetaPost\ 只会被编译一次。
\stopitemize

\noindent 大多数情况下,建议选用 uniqueMPgraphic,但若图形中存在一些需要每次使用时都要有所变化的内容,可选用 useMPgraphic。

另外需要注意,在 \ConTeXt\ 中使用 \MetaPost\ 时,通常会使用 \ConTeXt\ 定义的一些 \MetaPost\ 宏,这些宏构成的集合,名曰 \MetaFun。

\section{画一个盒子}

\MetaPost\ 作图语句遵守基本的英文语法,理解起来颇为简单。例如,用粗度为 2 pt 的圆头笔用暗红色绘制一条经过 $(0, 0)$,$(3\,{\rm cm}, 0)$,$(3\,{\rm cm}, 1\,{\rm cm})$,$(0, 1\,{\rm cm})$ 的封闭路径,

\starttyping[option=MP]
\startMPcode
pickup pencircle scaled 2pt;
draw (0, 0) -- (3cm, 0) -- (3cm, 1cm) -- (0, 1cm) -- cycle withcolor darkred;
\stopMPcode
\stoptyping
\startMPcode
pickup pencircle scaled 2pt;
draw (0, 0) -- (3cm, 0) -- (3cm, 1cm) -- (0, 1cm) -- cycle withcolor darkred;
\stopMPcode

上述代码中,\type{(0, 0) -- ... -- cycle} 构造的是一条封闭路径,可将其保存于一个路径变量:

\starttyping[option=MP]
path p;
p := (0, 0) -- (3cm, 0) -- (3cm, 1cm) -- (0, 1cm) -- cycle;
pickup pencircle scaled 2pt;
draw p withcolor darkred;
\stoptyping

将路径保存在变量中,是为了更便于对路径进行一些运算,例如

\starttyping[option=MP]
path p;
p := (0, 0) -- (3cm, 0) -- (3cm, 1cm) -- (0, 1cm) -- cycle;
pickup pencircle scaled 2pt;
draw p withcolor darkred;
draw p shifted (2cm, .5cm) withcolor darkblue;
\stoptyping
\startMPcode
path p;
p := (0, 0) -- (3cm, 0) -- (3cm, 1cm) -- (0, 1cm) -- cycle;
pickup pencircle scaled 2pt;
draw p withcolor darkred;
draw p shifted (2cm, .5cm) withcolor darkblue;
\stopMPcode
\noindent 路径 \type{p} 被向右平移了 2 cm,继而被向上平移动了 0.5 cm。

还有一种构造矩形路径的方法:先构造一个单位正方形,然后对其缩放。例如

\starttyping[option=MP]
pickup pencircle scaled 2pt;
draw fullsquare xscaled 3cm yscaled 1cm withcolor darkred;
\stoptyping
\startMPcode
pickup pencircle scaled 2pt;
draw fullsquare xscaled 3cm yscaled 1cm withcolor darkred;
\stopMPcode

可以使用 \MetaFun\ 宏 \type{randomized} 对路径进行随机扰动。例如,对一个宽为 3 cm,高为 1cm 的矩形路径以幅度 2mm 的程度予以扰动:

\starttyping[option=MP]
\startuseMPgraphic{随机晃动的矩形}
pickup pencircle scaled 2pt;
draw (fullsquare xscaled 3cm yscaled 1cm) randomized 2mm withcolor darkred;
\stopuseMPgraphic
\useMPgraphic{随机晃动的矩形}
\stoptyping
\startuseMPgraphic{随机晃动的矩形}
pickup pencircle scaled 2pt;
draw (fullsquare xscaled 3cm yscaled 1cm) randomized 2mm withcolor darkred;
\stopuseMPgraphic
\useMPgraphic{随机晃动的矩形}

还记得 overlay 吗?只要将上述 useMPgraphic 环境构造的图形制作为 overlay,便可将其作为 \type{\framed} 的背景,从而可以得到一种外观颇为别致的盒子。

\starttyping
\defineoverlay[晃晃][\useMPgraphic{随机晃动的矩形}]
\framed[frame=off,background=晃晃]{光辉岁月}
\stoptyping
\defineoverlay[晃晃][\useMPgraphic{随机晃动的矩形}]
\framed[frame=off,background=晃晃]{光辉岁月}

在 \ConTeXt\ 为 \MetaPost\ 提供的作图环境里,可分别通过 \type{\overlaywidth} 和 \type{\overlayheight} 获得 overlay 的宽度和高度。在将 overlay 作为 \type{\framed} 的背景时,\type{\framed} 的宽度和高度便是 overlay 的宽度和高度。基于这一特性,便可实现 \MetaPost\ 绘制的图形能够自动适应 \type{\framed} 的宽度和高度的变化。例如

\starttyping[option=MP]
\startuseMPgraphic{新的随机晃动的矩形}
path p;
p := fullsquare xscaled \overlaywidth yscaled \overlayheight;
pickup pencircle scaled 2pt;
draw p randomized 2mm withcolor darkred;
\stopuseMPgraphic

\defineoverlay[新的晃晃][\useMPgraphic{新的随机晃动的矩形}]
\framed
  [frame=off,background=新的晃晃]
  {今天只有残留的躯壳,迎接光辉岁月,风雨中抱紧自由。}
\stoptyping

\startuseMPgraphic{新的随机晃动的矩形}
path p;
p := fullsquare xscaled \overlaywidth yscaled \overlayheight;
pickup pencircle scaled 2pt;
draw p randomized 2mm withcolor darkred;
\stopuseMPgraphic

\defineoverlay[新的晃晃][\useMPgraphic{新的随机晃动的矩形}]
\framed[frame=off,background=新的晃晃]{今天只有残留的躯壳,迎接光辉岁月,风雨中抱紧自由。}

对于需要重复使用的盒子,为了避免每次重复设置其样式,可以将它定义为专用盒子。例如

\starttyping[option=TEX]
\defineframed[funnybox][frame=off,background=新的晃晃]
\funnybox{今天只有残留的躯壳,迎接光辉岁月,风雨中抱紧自由。}
\stoptyping

\MetaPost\ 可以为一条封闭路径填充颜色。在此需要明确,何为封闭路径。例如

\starttyping[option=MP]
path p, q, r;
p := (0, 0) -- (1, 0) -- (1, 1) -- (0, 0) -- (0, 0);
q := (0, 0) -- (1, 0) -- (1, 1) -- (0, 0) -- cycle;
r := fullsquare;
\stoptyping

\noindent 其中路径 \type{p} 的终点的坐标恰好是其起点,但它并非封闭路径,而路径 \type{q} 和 \type{r} 皆为封闭路径。下面示例,为封闭路径填充颜色:

\starttyping[option=MP]
path p;
p := (fullsquare xscaled 3cm yscaled 1cm) randomized 2mm;
pickup pencircle scaled 2pt;
fill p withcolor gray;
draw p withcolor darkred;
\stoptyping
\startMPcode
path p;
p := (fullsquare xscaled 3cm yscaled 1cm) randomized 2mm;
pickup pencircle scaled 2pt;
fill p withcolor darkgray;
draw p withcolor darkred;
\stopMPcode
\noindent 注意,对于封闭路径,应当先填充颜色,再绘制路径,否则所填充的颜色会覆盖一部分路径线条。